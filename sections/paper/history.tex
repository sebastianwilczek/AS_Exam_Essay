\section{History}
\label{sec:history}

iBeacon is a technological concept introduced by Apple, Inc. in 2013. An iBeacon is a (small) Bluetooth enabled device that was designed to be placed at static locations, with the goal to increase interaction with users of other Bluetooth enabled devices, for example customers in a technology store.

iBeacons were designed to make use of \ac{BLE}. \ac{BLE} was, in turn, introduced by Nokia in 2006, with first experiments happening in 2001. Originally, it had the name \textit{Wibree}. For marketability, it was called \textit{Bluetooth Smart} and was later integrated into the Bluetooth 4.0 specification, in early 2010.

\ac{BLE} made and still makes use of the same radio spectrum as regular Bluetooth, this being 2.400 GHz to 2.4835 GHz, with a total of 40 channels \cite{wiki}. The security of \ac{BLE} shifted slightly since its initial introduction as part of Bluetooth 4.0. Devices running Bluetooth 4.0 and 4.1 make use of what is now known as Legacy Pairing. The 4.2 specification introduced a slight change in the pairing process, making it slightly more secure. Since both the 4.2 as well as the legacy pairing are still in active use, please see \textit{\ref{sec:state} \nameref{sec:state}} for protocol and pairing details.

The pairing itself also evolved over time. The initial specification introduced three processes to create pairing: \textit{Just Works}, \textit{Out of Band}, and \textit{Passkey Entry}. With the change in the pairing protocol, another pairing method was introduced as part of specification 4.2: \textit{Numeric Comparison}. All of these pairing methods are still in use due to their varying degree of required interaction. Therefore, again, please see \textit{\ref{sec:state} \nameref{sec:state}} for implementational details.

iBeacons do make use of \ac{BLE}, however, in a less pairing-focused manner. The quirk of iBeacons is the fact that they operate entirely as advertisers. Every iBeacon or similarly implemented device has a unique identifier or UUID that is continuously transmitted using \ac{BLE}. \ac{BLE} is used due to its lower energy consumption, giving iBeacons a longer battery life, which is essential to static, non-interactive devices. This mode of operation has not changed since the initial introduction of iBeacon. The concept shows a slight comparability with \ac{NFC}. Both can have static, fixed location identifiers placed where required, to be read by an actively scanning device compatible with the technology. However, while \ac{NFC} has a rather limited range, iBeacons have a range of up to 70 meters, with some higher-powered devices reaching even beyond 400 meters.